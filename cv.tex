\documentclass[a4paper, 11pt]{moderncv}
\moderncvstyle{classic}
\moderncvcolor{orange}
\usepackage[utf8]{inputenc}

% adjust page margins
\usepackage[scale=0.8]{geometry}
\recomputelengths

% for language tables
\usepackage{booktabs}
\setlength\tabcolsep{6pt}

% for indenting tables
\usepackage{enumitem}
\setlist{itemindent=2.5mm}

\firstname{Giampaolo}
\familyname{Guiducci}
\title{\Large{\textsc{Lead Cybersecurity \break Architect}}}
\address{Roma, Italia}
\mobile{+39~349~3668324}
\email{giampaolo.guiducci@gmail.com}
\social[github]{https://github.com/gosub}

\begin{document}
\makecvtitle

\section{Esperienza professionale}
\cventry{2019--presente}{Lead Cybersecurity Architect}{NTT Data Italia S.p.a.}{Roma}{}
        {Ingegneria Applicativa e coordinazione tecnica di un gruppo di Gestione Applicativa per \emph{Telecom Italia} in ambito security:
\vspace{3mm}
%%%
          \begin{itemize}
          \item AAA OSC Radiator (Authentication, Authorization and Accounting)
          \item sonde IDS/IPS (Fortinet, Forcepoint, McAfee)
          \item DDOS Protection (Arbor Sightline, Arbor Insight)
          \item SIEM (McAfee ESM, Splunk Enterprise Security)
          \item monitoring (Nagios, Zabbix)
          \end{itemize}}
\vspace{3mm}
%%%
\cventry{2009--2019}{Analista Programmatore Senior}{Tecnorotoli S.r.l.}{Roma}{}
        {Progettazione e realizzazione di un sistema di cartellonistica informativa digitale per \emph{Policlinico Agostino Gemelli}.
          \begin{itemize}
          \item Creazione CMS dedicato con tecnologia PHP+MySQL in ambiente Centos Linux
          \item Implementazione server push, generazione programmatica della segnaletica e procedura di dispaccio temporizzata
          \item Sviluppo client embedded per unità di controllo video su single-board computer ARM
          \item Sviluppo interfaccia interattiva di ricerca e stampa percorsi per touchscreen informativi
          \item Realizzazione grafica HTML5/CSS, secondo le linee guida del cliente
          \item Installazione del server su piattaforma VMware
          \end{itemize}
        \vspace{10pt}
          Progettazione e realizzazione di un impianto di chiamata vocale digitalizzato per \emph{Pronto Soccorso Policlinico Agostino Gemelli}.
          \begin{itemize}
          \item Sviluppo server e client di chiamata con tecnologia PHP+MySQL+XUL su piattaforma GNU/Linux
          \item Realizzazione delle registrazioni attraverso sintetizzatore vocale
          \item Installazione di display dedicati per la visualizzazione della chiamata
          \item Installazione impianto audio di amplificazione e riproduzione nella zona triage e sala d'attesa, con integrazione dell'impianto microfonico preesistente
          \end{itemize}}
%%%
\vspace{3mm}
%%%
\cventry{2004--2015}{Analista Programmatore}{Theta S.r.l.}{Roma}{}
        {Realizzazione di impianti per la gestione delle attese per privati, aziende ospedaliere e pubblica 
          amministrazione su tutto il territorio nazionale, tra cui: \emph{ACI Automobile Club d'Italia}, \emph{ASL}, \emph{INPS}, \emph{Policlinico Agostino Gemelli}, \emph{Comune di Roma Ufficio del Condono Edilizio}, \emph{Ospedale San Carlo di Nancy}, \emph{IFO Istituto Regina Elena}, \emph{Ospedale Nuovo Regina Margherita}, \emph{Equitalia}, \emph{Presidio Ospedaliero Oftalmico}, \emph{INMI Lazzaro Spallanzani}.
          \vspace{10pt}
          Tra le mansioni svolte:
          \begin{itemize}
          \item Sopralluoghi e raccolta delle specifiche del cliente
          \item Progettazione dell'impianto con hardware dedicato e software in-house
          \item Realizzazione e installazione hardware e software
          \item Istruzione e formazione del personale
          \item Assistenza telefonica e on-site
          \item Interventi di riparazione e manutenzione
          \item Gestione del rapporto tecnico con il cliente dalle fasi iniziali al completamento del progetto
          \end{itemize}
          \vspace{10pt}
          Customizzazione della componente software degli impianti di gestione code, adattandola alle esigenze del cliente. Tra le altre:
          Per IDI – Istituto dermopatico dell'Immacolata
          \begin{itemize}
          \item Migrazione dell'impianto da ambiente Windows ad ambiente Linux, e successivamente alla virtualizzazione dei client
          \item Integrazione con sistema di report preesistente
          \item Sviluppo di feature per l'automazione, l'analisi e lo snellimento delle procedure di sportello
          \end{itemize}
          Per CPI – Centri per l'impiego
          \begin{itemize}
          \item Sviluppo di API dedicate per l'integrazione con applicazioni Android
          \item Sincronizzazione dell'orologio di sede via internet
            Per GORI acqua
          \item Raccolta e integrazione da remoto delle statistiche di afflusso
        \end{itemize}}
%%%
\vspace{3mm}
%%%
\cventry{2004--2015}{Sistemista}{Theta S.r.l.}{Roma}{}
        {\begin{itemize}
          \item Gestione della rete aziendale e del parco PC
          \item Amministrazione dei server aziendali, fisici e virtuali (VPS): web, mail, DNS, NAS, backup
          \item Registrazione e mantenimento di domini internet presso il \emph{Registro IT} in qualità di Maintainer e Registrar (protocollo EPP)
          \item Responsabile delle procedure di backup e dei sistemi di sicurezza informatica
          \item Sviluppo di un sistema per l'accesso protetto ad internet per operazioni di online banking, basato su GNU/Linux
        \end{itemize}}
%%%
\cventry{2002--2004}{Programmatore}{Turnomatic S.r.l}{Roma}{}{Sviluppo interfaccia utente desktop per sistema di allarme per operatori di sportello e casse (Visual Basic 5)}
%%%        
\cventry{2000--2001}{Apprendista Programmatore}{RP Elettronica S.r.l}{Roma}{}{Partecipazione allo sviluppo del firmware di un display informativo a led (C, PIC assembly)}
%%%
\section{Istruzione}
\cventry{1996--2000}{Maturità Scientifica}{L.S.S. Amedeo Avogadro}{Roma}{\textit{96/100}}{}
%%%
\section{Lingue}
%%%
\cvlanguage{Italiano}{madrelingua}{}
\vspace{5pt}
\cvlanguage{Inglese}{%
  \centering
  \begin{tabular}{*{5}{c}}
    \toprule
    \multicolumn{2}{c}{COMPRENSIONE} & \multicolumn{2}{c}{PARLATO}          & PRODUZIONE \\
    &               &                &                                      & SCRITTA    \\ \midrule
    Ascolto         & Lettura        & Interazione & Produzione orale & -                \\ \hline
    C2              & C2             & C1          & C2                     & C1         \\ \bottomrule
  \end{tabular}%
}{}
%%%
\section{Competenze personali}
%%%
\subsection{Competenze comunicative}
\cvline{}{Buone competenze comunicative acquisite durante l’esperienza professionale nei seguenti ambiti:
\begin{itemize}
\item Raccolta dei requisiti utente
\item Proposta di soluzioni tecniche al cliente
\item Formazione del personale per utilizzo software
\item Assistenza e supporto tecnico
\item Produzione di documentazione scritta
\end{itemize}}
\subsection{Competenze organizzative e gestionali}
\cvline{}{Capacità di organizzare e coordinare gruppi di lavoro acquisita in ambito professionale. Coordinamento di un team di professionisti in fase di progettazione, messa in opera e revisione di progetti tecnici}
\subsection{Competenze professionali}
\cvline{}{\begin{itemize}
\item Capacità di analisi e implementazione di soluzioni informatiche
\item Conoscenza e applicazione dei paradigmi di programmazione procedurale, orientata agli oggetti, funzionale e logica
\item Progettazione di architetture informatiche con tecnologia LAMP
\item Sviluppo di soluzioni embedded su dispositivi integrati
\item Progettazione di basi dati
\item Gestione reti e server internet e intranet
\item Analisi, progettazione e implementazione di applicazioni in ambiente Android
\item Sviluppo grafico e multimediale, fotoritocco, fotografia digitale
\item Webdesign e impaginazione
\item Adattabilità a nuove piattaforme HW/SW, ambienti operativi, linguaggi di programmazione
\end{itemize}}
\subsection{Competenza digitale}
\cvline{}{%
  \centering
  \begin{tabular}{*{5}{c}}
    \toprule
    \multicolumn{5}{c}{AUTOVALUTAZIONE} \\ \hline
    Elaborazione        & Comunicazione & Creazione    & Sicurezza & Risoluzione  \\
    delle informazioni  &               & di contenuti &           & dei problemi \\ \midrule
    Avanzato            & Avanzato      & Avanzato     & Avanzato  & Avanzato     \\ \bottomrule
  \end{tabular}%
}
%%%
\subsection{Competenze informatiche}
\cvcomputer{Linguaggi}{PHP, Python, Erlang, Javascript, C, VB5, Lisp, Haskell}{S.O.}{Linux, Windows, MSDOS, Android}
\cvcomputer{Scripting}{Bash, AWK, VBA}{Versioning}{Git, Bazaar}
\cvcomputer{Markup}{HTML5, XML, CSS, SVG, PostScript, \LaTeX}{Build}{GNU Make}
\cvcomputer{WebServer}{Apache}{Backup}{FreeNAS, rsnapshot}
\cvcomputer{VM}{VirtualBox, WMWare Vsphere}{DBMS}{MySQL, MariaDB, SQLite}
\cvcomputer{Containers}{Docker, Podman}{Config}{Ansible}
\cvcomputer{Graphics}{Inkscape, GIMP, Adobe Photoshop, ImageMagick}{Distros}{Centos, Redhat, Debian, Arch, Slackware}
\cvcomputer{Browsers}{Chrome, Firefox, Edge, IE}{Mail}{Outlook, Thunderbird, Evolution}
\cvcomputer{Office}{Microsoft Office, Open/Libre Office}{}{}
%%%
\subsection{Altre competenze}
\cvline{}{\begin{itemize}
  \item Ottima capacità di analisi logico matematica.
  \item  Ottima capacità di individuazione e risoluzione problemi
\end{itemize}}
%%%
\section{Ulteriori Informazioni}
\cvline{Patente di guida}{B}
\subsection{Certificazioni}
\cvline{2023--2026}{(ISC)$^2$ CCSP -- Certified Cloud Security Professional}
\cvline{2023--2026}{CompTIA Security+}
\cvline{2015}{Abilitazione alla gestione tecnica ai sensi del D.M. 37/2008 per l'esercizio delle attività di cui alla lettera A, B, C, D, E}
\subsection{Corsi}
\cvline{2017}{Corso DNS Avanzato per Maintainer organizzato da \emph{Registro .IT}}
\cvline{2017}{Corso DNS Base per Maintainer organizzato da \emph{Registro .IT}}
% burocrazia
\emptysection{}\closesection
\vfill
\begin{center}
\textit{\small Autorizzo il trattamento dei miei dati personali ai sensi del Decreto Legislativo 30 giugno 2003, n. 196 ``Codice in materia di protezione dei dati personali''.}
\end{center}

\end{document}
