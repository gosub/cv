\section{Esperienza professionale}
\cventry{2024--\footnotesize{presente}}{Lead Cybersecurity Architect}{NTT Data Italia S.p.a.}{Roma}{}
        {Progettazione, provisioning e gestione di piattaforme SAST/DAST/SCA, con integrazione in pipeline di sviluppo DevSecOps, per importanti operatori del settore bancario a livello nazionale. Tra le attività svolte: \begin{itemize}
          \item Sviluppo e supporto pipeline DevSecOps
          \item Integrazione in GitLab, Github Enterprise, Jenkins
          \item Sviluppo di tool customizzati di integrazione
          \item Collaborazione e intermediazione con i team di sviluppo
          \item Efficientamento ed evoluzione delle piattaforme
          \end{itemize}}
%%%
\vspace{3mm}
%%%
\cventry{2021--2024}{Lead Cybersecurity Architect}{NTT Data Italia S.p.a.}{Roma}{}
        {Coordinatore di un gruppo di Gestione Applicativa per \emph{Telecom Italia} in ambito Security.\\ Tra le applicazioni gestite: \begin{itemize}
          \item AAA (Authentication, Authorization and Accounting)
          \item sonde IDS/IPS
          \item DDOS Protection
          \item SIEM
          \item SOAR
          \item Proxy web aziendale
          \end{itemize}
          Amministrazione sistemistica e applicativa. Applicazione delle policy aziendali e best practice di cybersecurity e information security. Gestione del lifecycle applicativo, monitoraggio e availability.}
%%%
\vspace{3mm}
%%%
\cventry{2019--2021}{Senior Cybersecurity Engineer}{NTT Data Italia S.p.a.}{Roma}{}
        {Ingegneria Applicativa per \emph{Telecom Italia} in ambito Security. \\ Progetti seguiti:
          \begin{itemize}
          \item AAA (Authentication, Authorization and Accounting)
          \item sonde IDS/IPS
        \end{itemize}
          Progettazione e realizzazione soluzioni di cybersecurity.\\
          Produzione di documentazione tecnica.}
%%%
\vspace{3mm}
%%%
\cventry{2009--2019}{Analista Programmatore Senior}{Tecnorotoli S.r.l.}{Roma}{}
        {Progettazione, sviluppo, installazione e manutenzione per \emph{Policlinico Agostino Gemelli} di:
          \begin{itemize}
          \item  sistema di cartellonistica informativa digitale
          \item  impianto di gestione code con chiamata audio vocale digitalizzato
          \end{itemize}
          Tra le attività svolte:
          \begin{itemize}
          \item Creazione CMS dedicato con stack LAMP (Linux, Apache, MySQL, PHP)
          \item Sviluppo client embedded per unità di controllo video su single-board computer ARM
          \item Sviluppo interfaccia interattiva di ricerca e stampa percorsi per touchscreen informativi
          \item Realizzazione grafica HTML5/CSS, secondo le linee guida del cliente
          \item Installazione su piattaforma VMware
          \end{itemize}}
%%%
\vspace{3mm}
%%%
\cventry{2004--2015}{Analista Programmatore}{Theta S.r.l.}{Roma}{}
        {Realizzazione di impianti per la gestione delle attese per privati, aziende ospedaliere e pubblica 
          amministrazione su tutto il territorio nazionale, tra cui: \emph{ACI Automobile Club d'Italia}, \emph{ASL}, \emph{INPS}, \emph{Policlinico Agostino Gemelli}, \emph{Comune di Roma Ufficio del Condono Edilizio}, \emph{Ospedale San Carlo di Nancy}, \emph{IFO Istituto Regina Elena}, \emph{Ospedale Nuovo Regina Margherita}, \emph{Equitalia}, \emph{Presidio Ospedaliero Oftalmico}, \emph{INMI Lazzaro Spallanzani}.\\Tra le mansioni svolte:
          \begin{itemize}
          \item Gestione del rapporto tecnico con il cliente in tutte le fasi del progetto
          \item Sopralluoghi e raccolta delle specifiche di progetto
          \item Progettazione dell'impianto con hardware dedicato e software in-house
          \item Customizzazione della componente software secondo le esigenze del cliente
          \item Realizzazione e installazione hardware e software
          \item Istruzione e formazione del personale
          \item Assistenza telefonica e on-site
          \item Interventi di riparazione e manutenzione
          \end{itemize}}
%%%
\vspace{3mm}
%%%
\cventry{2004--2015}{Sistemista}{Theta S.r.l.}{Roma}{}
        {\begin{itemize}
          \item Amministrazione della rete aziendale e del parco PC
          \item Gestione dei server aziendali, fisici e virtuali (VPS): web, mail, DNS, NAS, backup
          \item Registrazione e mantenimento di domini internet presso il \emph{Registro IT} in qualità di Maintainer e Registrar (protocollo EPP)
          \item Responsabile delle procedure di backup e dei sistemi di sicurezza informatica
          \item Sviluppo di un sistema per l'accesso protetto ad internet per online banking, basato su GNU/Linux
        \end{itemize}}
%%%
\section{Istruzione}
\cventry{1995--2000}{Maturità Scientifica}{L.S.S. Amedeo Avogadro}{Roma}{\textit{96/100}}{}
%%%
\section{Certificazioni}
\cvline{2023}{(ISC)$^2$ CCSP -- Certified Cloud Security Professional}
\cvline{2023}{CompTIA Security+}
%%%
\section{Lingue}
%%%
\cvlanguage{Italiano}{madrelingua}{}
\vspace{3mm}
\cvlanguage{Inglese}{%
  \centering
  \begin{tabular}{*{5}{c}}
    \toprule
    \multicolumn{2}{c}{Comprensione} & \multicolumn{2}{c}{Parlato}          & Scritto \\ \midrule
    Ascolto         & Lettura        & Interazione & Produzione orale & -             \\
    C2              & C2             & C1          & C2                     & C1      \\ \bottomrule
    \multicolumn{5}{c}{\tiny{Livelli: A1/2 Livello base - B1/2 Livello intermedio - C1/2 Livello avanzato}}\\
    \multicolumn{5}{c}{\tiny{Quadro Comune Europeo di Riferimento delle Lingue}}\\
  \end{tabular}%
}{}
%%%
\closesection
\pagebreak
%%%
\section{Competenze professionali}
%%%
\cvlistitem{Team Leading and Coordination}
\cvlistitem{Project Management}
\cvlistitem{IT Security Administration}
\cvlistitem{IT Security Architecture Design (data protection, security operation center, identity \& access management, electronic signature, certification authority)}
\cvlistitem{Sviluppo e scripting (conoscenza e applicazione dei paradigmi di programmazione procedurale, orientata agli oggetti, funzionale e logica)}
\cvlistitem{Secure DevOps}
\cvlistitem{Spiccate capacità comunicative}
\cvlistitem{Ottima capacità di analisi logico matematica}
\cvlistitem{Ottima capacità di individuazione e risoluzione problemi}
\cvlistitem{Produzione di documentazione tecnica}
\cvlistitem{Naturale predisposizione alla formazione}
\cvlistitem{Fondamenti di grafica, fotoritocco, impaginazione, webdesign}
%%%
\section{Competenze tecniche}
%%%
\cvlistitem{Intrusion Prevention/Detection (Fortinet, Forcepoint NGFW, McAfee NSM)}
\cvlistitem{AAA (OSC Radiator, Cisco ACS)}
\cvlistitem{DDOS Protection (Arbor Sightline, Arbor Insight)}
\cvlistitem{SIEM (McAfee ESM, Splunk Enterprise Security)}
\cvlistitem{Secure Proxy (Trellix WebGateway, Squid)}
\cvlistitem{Monitoring (Nagios, Zabbix)}
\cvlistitem{LDAP (OpenDJ)}
\cvlistitem{DBMS (MySQL, MariaDB, SQLite)}
\cvlistitem{Linguaggi (Python, PHP, Javascript, C, VB5, Lisp, Erlang, Haskell)}
\cvlistitem{Scripting (Bash, AWK, VBA)}
\cvlistitem{Sistemi Operativi (Linux, Windows, MSDOS, Android)}
\cvlistitem{Distribuzioni GNU/Linux (Centos, Red Hat, Debian, Arch, Slackware)}
\cvlistitem{Web Server (Apache)}
\cvlistitem{Strumenti di Backup (FreeNAS, rsnapshot)}
\cvlistitem{Virtualizzazione e Containers (Docker, Podman, VirtualBox, VMware vSphere)}
\cvlistitem{Versioning (Git, Bazaar)}
\cvlistitem{Linguaggi di Markup (HTML5, XML, CSS, SVG, PostScript, \LaTeX)}
\cvlistitem{Build Systems (GNU Make)}
\cvlistitem{Config Management (Ansible)}
\cvlistitem{Grafica (Inkscape, GIMP, Adobe Photoshop, ImageMagick)}

%%%
\section{Interessi}
\cvlistitem{Linguaggi di programmazione, Algoritmi, Computer Science}
\cvlistitem{Sintesi audio, DSP, Livecoding}
\cvlistitem{Machine Learning, Deep Learning}
%%%
\section{Ulteriori Informazioni}
\cvline{\scriptsize{Patente di guida}}{B}
\subsection{Abilitazioni}
\cvline{2015}{Abilitazione alla gestione tecnica ai sensi del D.M. 37/2008 per l'esercizio delle attività di cui alla lettera A, B, C, D, E}
\subsection{Corsi}
\cvline{2017}{Corso DNS Avanzato per Maintainer organizzato da \emph{Registro .IT}}
\cvline{2017}{Corso DNS Base per Maintainer organizzato da \emph{Registro .IT}}
% burocrazia
\emptysection{}\closesection
\vfill
\begin{center}
\textit{\small Autorizzo il trattamento dei miei dati personali ai sensi del Decreto Legislativo 30 giugno 2003, n. 196 ``Codice in materia di protezione dei dati personali''.}
\end{center}
